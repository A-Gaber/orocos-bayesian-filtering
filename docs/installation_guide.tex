\documentclass[a4paper,10pt]{article}

\advance\textwidth by 6cm
\advance\oddsidemargin by -3cm
\advance\evensidemargin by -3cm

\advance \topmargin -2cm
\advance \textheight 4cm


%opening
\title{Installation Guide \\  Bayesian Filtering Library}
\author{Tinne De Laet, Wim Meeussen, Klaas Gadeyne}

\usepackage{amsmath}
\usepackage{subfigure}
\usepackage{graphicx,color,psfrag}
\usepackage{epsfig} 
\usepackage{fancyvrb} 
\usepackage{fancybox} 
\usepackage{verbatim}
\usepackage{keyval}
\usepackage{url}
%\usepackage[]{tex4ht}

%\graphicspath{{figures_src/}}

\begin{document}

\maketitle



%----------------------------------
\section{Precompiled Ubuntu/Debian packages}
%----------------------------------
\label{sec:ubuntu}
For Ubuntu and Debian users, the easiest way to install BFL is by
using our precompiled packages.  The BFL package is compiled for the
LTI matrix library and the LTI random number generator library.  The
repository also includes a precompiled version of the LTI library
itself.  Using these precompiled packages is very simple:
\begin{enumerate}
\item First add the following line to your /etc/apt/sources.list:
\begin{verbatim}
  deb http://people.mech.kuleuven.be/~tdelaet/bfl _mydistro_ main
\end{verbatim}
  where \emph{\_mydistro\_} is breezy, dapper, edgy or feisty.
\item Then apt-get the packages:
\begin{verbatim}
  sudo apt-get update
  sudo apt-get install bfl
\end{verbatim}
\end{enumerate}
Now all components are installed and you are ready to start using BFL.




%----------------------------------
\section{Installation from source}
%----------------------------------
An alternative way to get BFL is to install it from source, e.g.~if
you are not using Ubuntu or Debian. This section gives step-by-step
instructions to compile BFL from source. You'll not only need to
compile BFL, but also a matrix computation library and a random number
generator library. A few things you need to know before getting
started:
\begin{enumerate}
\item BFL is developed and regularly tested for GNU Linux on
  Debian and Ubuntu. However we've received success reports from other
  distributions and architectures, such as Gentoo, MAC OS X and MS
  Windows.
\item BFL compiles without warnings with g++ 2.95.0 up to g++ 4.0.
  However, it is preferred to use the latest g++ version.
\item We use cmake in the build system of BFL. You'll need cmake 2.2
  or 2.4.
\item To download the latest version from our subversion server, you
  will need a subversion client.
\end{enumerate}
The instructions explain how to install libraries in the /usr/local
directory. You will need write permissions in this directory. When you
don't have these permissions, you can also install everything in your
home directory.

\subsection{Install an external matrix library}
%----------------------------------
BFL requires an external library for matrix computations. You can
choose to use the LTI library, or the Newmat library.

\subsubsection{LTI matrix library}
\label{subsec:lti}
You can install the LTI library as a precompiled Ubuntu/Debian
package, or compile it from source. 
\begin{itemize}
\item To install the precompiled LTI package, add the line in your
  sources.list as described in section~\ref{sec:ubuntu}, and then run:
\begin{verbatim}
  sudo apt-get update
  sudo apt-get install liblti-dev
\end{verbatim}
\item To install the LTI library from source:

\begin{enumerate}
\item First you'll need to install the libxt-dev package:
\begin{verbatim}
  sudo apt-get install libxt-dev
\end{verbatim}
\item Then download the latest version of the LTI library from the LTI
  web page
  \url{http://ltilib.sourceforge.net/doc/homepage/index.shtml}
\item Move the downloaded file to a sources folder:
\begin{verbatim}
  mv ltilib.tar.gz ~/sources/
\end{verbatim}
\item Go to the directory \ $\mathtt{\tilde{ }}$\texttt{/sources/}
\begin{verbatim}
  cd ~/sources/
\end{verbatim}
\item Untar and unzip the file:
\begin{verbatim}
  tar -xzvf ltilib.tar.gz
\end{verbatim}
\item Build the library:
\begin{verbatim}
  cd ltilib/linux
  make -f Makefile.cvs
  ./configure --disable-debug --without-gtk --disable-gtk
  make
\end{verbatim}
  If you don't want to install ltilib in \texttt{/usr/local}
  (e.g.~because you don't have admin rights), pass the
  \texttt{--prefix} option to the configure script.
\item Finally, install the library
\begin{verbatim}
  [sudo] make install
\end{verbatim}
\end{enumerate}
For more details, see the LTI web page
\url{http://ltilib.sourceforge.net/doc/homepage/index.shtml}.
\end{itemize}






\subsubsection{Newmat matrix library}
You can install the Newmat library as a precompiled Ubuntu/Debian
package, or compile it from source.
\begin{itemize}
\item To install the precompiled Newmat package run:
\begin{verbatim}
  sudo apt-get install libnewmat10-dev
\end{verbatim}


\item To install the Newmat library from source:
\begin{enumerate}
\item Create a directory in which you will put the source of newmat:
\begin{verbatim}
 mkdir ~/sources/newmat/
\end{verbatim}
\item Then download the latest version (version 11) of the Newmat library from the
  Newmat web page:
  \url{http://www.robertnz.net/download.html}
\item Move the downloaded file to your source folder:
\begin{verbatim}
  mv newmat11.tar.gz ~/sources/newmat
\end{verbatim}
\item Go to the directory \ $\mathtt{\tilde{ }}$\texttt{/sources/}
\begin{verbatim}
  cd ~/sources/newmat
\end{verbatim}
\item Untar and unzip the file:
\begin{verbatim}
  tar -xzvf newmat11.tar.gz
\end{verbatim}
\item Then go to \url{http://people.mech.kuleuven.be/~tdelaet/Makefile.Newmat} and save the file as Makefile in your newmat directory.
\item Also get the adapted include.h at \url{http://people.mech.kuleuven.be/~tdelaet/include.h.newmat11} and save the file as include.h. in your newmat directory.
\item Build the library:
\begin{verbatim}
  make 
\end{verbatim}
\item Finally, install the library.  By default newmat gets installed
  in \texttt{/usr/local}.  If you want to install it somewhere else,
  edit the \emph{PREFIX} variable in \texttt{Makefile.newmat} before
  running make install.
\begin{verbatim}
  [sudo] make install
\end{verbatim}
\end{enumerate}
\end{itemize}








\subsection{Install an external random number generator library}
%----------------------------------
BFL requires an external library for matrix computations. You can
choose to use the LTI library, or the Boost library.

\subsubsection{LTI rng library}
You can install the LTI library as a precompiled Ubuntu/Debian
package, or compile it from source (follow the same instructions as
described in section~\ref{subsec:lti}). If you already installed LTI
as the matrix library, you already have everything for the rng library.

\subsubsection{Boost rng library}
You can install the Boost library as a precompiled Ubuntu/Debian
package, or install it from source.
\begin{itemize}
\item To install the precompiled Boost package run:
\begin{verbatim}
  sudo apt-get install libboost-dev
\end{verbatim}
\item To get the source of the Boost library from source, download it from \url{http://sourceforge.net/project/showfiles.php?group_id=7586}.
\begin{enumerate}
 \item Move the downloaded file to your source folder:
	\begin{verbatim}
	mv boost_1_34_0.tar.gz ~/sources/
	\end{verbatim}
 \item Untar and unzip the file?
 	\begin{verbatim}
 	 tar -xzvf boost_1_34_0.tar.gz
 	\end{verbatim}
 \item Go into the created directory:
 	\begin{verbatim}
 	cd boost_1_34_o
 	\end{verbatim}
 \item Configure the library:
	\begin{verbatim}
	 ./configure
	\end{verbatim}
 \item Make the package:
	\begin{verbatim}
	 make
	\end{verbatim}
\item Install boost:
	\begin{verbatim}
	 sudo make install
	\end{verbatim}
\end{enumerate}

\end{itemize}







\subsection{Install BFL itself}
%----------------------------------
Now we arrived to the installation of the BFL library itself. BFL is
available from our subversion server.
% FIXME: Provide installation instructions for releases too!
\begin{enumerate}
\item First go to the \ $\mathtt{\tilde{ }}$\texttt{/sources/} folder:
\begin{verbatim}
  cd ~/sources/
\end{verbatim}
\item Then, to get a copy of BFL, use:
\begin{verbatim}
  svn co http://svn.mech.kuleuven.be/repos/orocos/trunk/bfl
\end{verbatim}
\item Now create and go to a build directory in ~/sources/bfl
\begin{verbatim}
  mkdir ~/sources/bfl/build
  cd ~/sources/bfl/build
\end{verbatim}
\item Run ccmake:
\begin{verbatim}
  ccmake ..
\end{verbatim}
\item Then type 'c' to configure, and 'e' to exit the page that shows
  the configure output. Now you see the cmake configuration menu. In
  this menu you can change the following options:
\begin{itemize}
\item matrix\_install: the path where you external matrix library is
  installed
\item matrix\_lib: the matrix library you choose: lti or newmat
\item rng\_install: the path where you external rng library is
  installed
\item rng\_lib: the rng library you choose: lti or boost
\end{itemize}
Now repeat the 'c' configure and 'e' exit cycle, until you get no more
errors. When this is the case, you'll have the 'g' generate option
available. Press 'g' to generate the makefiles, and then quit cmake
with 'q' quit.
\item Now all configuration is done, and you can build BFL:
\begin{verbatim}
  make 
\end{verbatim}	
\item To check the functionality of BFL, use
\begin{verbatim}
  make check
\end{verbatim}	
\item To install BFL use
\begin{verbatim}
  sudo make install
\end{verbatim}
\end{enumerate}
In the \ $\mathtt{\tilde{ }}$\texttt{/sources/sources/bfl/examples} directory, you find some example BFL
filters. A good next step is to check out the BFL tutorial on
\url{http://www.orocos.org/bfl}, for a step-by-step introduction in building
your own filters in BFL.
\\\\\\
Good luck!


\end{document}

% LocalWords:  doxygen svn
